\documentclass{article}
\usepackage{graphicx}
\usepackage{float}
\usepackage{qtree}
\usepackage{sidecap}
\usepackage{fullpage}
\usepackage{listings}
\usepackage{color}
\usepackage{tikz}
\usetikzlibrary{shapes,arrows, matrix, positioning, fit}
\DeclareGraphicsExtensions{.pdf,.png,.jpg}
\graphicspath{ {img/} }

% Java code with lstlisting
\definecolor{dkgreen}{rgb}{0,0.6,0}
\definecolor{gray}{rgb}{0.5,0.5,0.5}
\definecolor{mauve}{rgb}{0.58,0,0.82}

\lstset{frame=tb,
    language=Java,
    aboveskip=3mm,
    belowskip=3mm,
    showstringspaces=false,
    columns=flexible,
    basicstyle={\small\ttfamily},
    numbers=none,
    numberstyle=\tiny\color{gray},
    keywordstyle=\color{blue},
    commentstyle=\color{dkgreen},
    stringstyle=\color{mauve},
    breaklines=true,
    breakatwhitespace=true
    tabsize=3
}

\title{CIS4301 Notes}
\author{Ryan Roden-Corrent}
\date{Wed Jan 22 10:49:19 EST 2014}

\begin{document}
\setlength\parindent{0pt}
% Tikz general settings
\tikzstyle{relation} = [diamond, draw, fill=blue!20, text width=4em,
  text badly centered, node distance=3cm, inner sep=0pt]
\tikzstyle{attribute} = [draw, ellipse, fill=red!20, node distance=2.5cm,
  minimum height=2em]
\tikzstyle{entity} = [rectangle, draw, fill=blue!20, text width=5em,
  text centered, minimum height=4em]
\tikzstyle{relation-weak} = [diamond, double, draw, fill=blue!20, text width=4em,
  text badly centered, node distance=3cm, inner sep=0pt]
\tikzstyle{entity-weak} = [rectangle, draw, double, fill=blue!20, text width=5em,
  text centered, minimum height=4em]
\tikzstyle{line} = [draw, -]
\tikzstyle{arrow} = [draw, -latex']
\tikzstyle{arrow-round} = [draw, -), thick]
\maketitle

\section{Entity Relationship (ER) Models}
\subsection{Example ER Model}

\begin{figure}[H]
\begin{tikzpicture}[node distance = 2cm, auto]
  \node [entity-weak] (logins) {logins};
  \node [attribute, left of=logins] (billing) {billing};
  \node [attribute, above of=logins] (name) {\underline{name}};
  \node [relation-weak, right of=logins] (at) {at};
  \node [entity, right of=at, node distance=3cm] (host) {host};
  \node [attribute, above of=host] (hostname) {\underline{name}};
  \node [attribute, right of=host] (url) {url};

  \path [line] (billing) -- (logins);
  \path [line] (name) -- (logins);

  \path [line] (logins) -- (at);
  \path [arrow] (at) -- (host);

  \path [line] (hostname) -- (host);
  \path [line] (url) -- (host);
\end{tikzpicture}
\end{figure}

Host is a \textbf{strong entity} (can exist on its own)\\
Login is a \textbf{weak entity} (cannot exist without Host)\\
\begin{description}
  \item[Host] uniquely identified by name
  \item[name] underlined - key for Host (must be unique)
  \item[URL]
  \item[At] Circular arrow - maps to a single Host, cannot exist without it
  \item[Logins] A login requires a Host
  \item[name] underlined - key for name (must be unique)
  \item[Host]
\end{description}

\subsection{Schema Format}
\begin{itemize}
  \item{Host(\underline{name}, URL)}
  \item{Logins(\underline{name}, \underline{host.name}), billing}
\end{itemize}

\pagebreak
\section{ER Diagrams}
\begin{figure}[h!]
  \centering
  \caption{Example ER Diagram}
\begin{tikzpicture}[node distance = 2cm, auto]
  \node [entity] (stars) {stars};
  \node [attribute, above of=stars] (name) {\underline{name}};
  \node [attribute, above left of=stars] (DoB) {\underline{DoB}};
  \node [attribute, left of=stars] (other) {\ldots};
  \node [relation, right of=stars] (contracts) {contracts};
  \node [attribute, above of=contracts] (salary) {salary};
  \node [entity, right =1.5cm of contracts] (movies) {movies};
  \node [attribute, above of=movies] (title) {\underline{title}};
  \node [entity, below of=contracts] (studios) {studios};
  \node [attribute, left of=studios] (room) {\underline{room}};

  \path [line] (stars) -- (name);
  \path [line] (stars) -- (DoB);
  \path [line] (stars) -- (other);
  \path [line] (stars) -- (contracts);
  \path [line] (contracts) -- (movies);
  \path [line] (contracts) -- (salary);
  \path [line] (title) -- (movies);
  \path [line] (studios) -- (room);

  \path [arrow] (contracts) -- (studios);

\end{tikzpicture}
\end{figure}
\begin{description}
    \item[stars-movies]{many-many}
    \item[stars-studios]{many-one}
    \item[movies-studios]{many-one}
\end{description}
An \underline{underlined attribute} indicates a key for the entity. If there
are no underlined attributes for an entity, the entire collection of attributes
is the unique identifier for that entity.\\
A star can only be associated with one studio, but a studio can be associated
with many stars.\\
A movie can only be associated with one studio, but a studio can be associated
with many movies.\\\\
\textbf{Schema:}
\begin{itemize}
    \item{stars(\underline{name}, \underline{DoB}, \ldots)}
    \item{stars(\underline{title}, )}
    \item{Contracts(\underline{StudioName}, \underline{starname},
        \underline{DoB}, \underline{movie-title}, salary)}
\end{itemize}
%MAKE THIS A TABLE
%
Example Contracts:\\
\begin{itemize}
    \item{(Ricky R, Jaws, MGM)}
    \item{(Ricky R, Batman, WB)}
    \item{(Michael Keaton, Batman, WB)}
\end{itemize}
StudioName, StarName, and MovieTitle are all required to uniquely identify a
contract.

\subsection{Multiple Key Example}
\begin{figure}[H]
  \centering
  \caption{A car is uniquely identified by the tuple (make,model,color,year)}
  \begin{tikzpicture}[node distance = 2cm, auto]
    \node [entity] (car) {car};
    \node [attribute, above of=car] (make) {make};
    \node [attribute, above left of=car] (model) {model};
    \node [attribute, above right of=car] (color) {color};
    \node [attribute, right of=car] (year) {year};
    \path [line] (car) -- (make);
    \path [line] (car) -- (model);
    \path [line] (car) -- (color);
    \path [line] (car) -- (year);
  \end{tikzpicture}
\end{figure}
\subsection{Weak Entity Example}
\begin{figure}[H]
  \centering
  \caption{A car cannot exist without a VIN issued by a Dealer}
  \begin{tikzpicture}[node distance = 2cm, auto]
    \node [entity-weak] (car) {car};
    \node [attribute, above of=car] (make) {make};
    \node [attribute, above left of=car] (model) {model};
    \node [attribute, above right of=car] (color) {color};
    \node [attribute, right of=car] (year) {year};
    \node [relation-weak, left=2cm of car] (issues) {Issues};
    \node [entity, left=2cm of issues] (dealer) {dealer};
    \node [attribute, above of=issues] (VIN) {VIN};
    \path [line] (car) -- (make);
    \path [line] (car) -- (model);
    \path [line] (car) -- (color);
    \path [line] (car) -- (year);
    \path [line] (car) -- (issues);
    \path [line] (issues) -- (VIN);
    \path [arrow] (issues) -- (dealer);
  \end{tikzpicture}
\end{figure}
Issues(\underline{VIN},\underline{Dealer})

\subsection{Another Example}
\begin{figure}[H]
  \centering
  \caption{Example ER Diagram}
\begin{tikzpicture}[node distance = 2cm, auto]
  \node [entity] (movies) {movies};
  \node [relation, right of=movies] (owns) {owns};
  \node [entity, right of = owns] (studio) {studio};
  \node [relation, right of=studio] (runs) {runs};
  \node [entity, right of = runs] (president) {president};

  \path [line] (movies) -- (owns);
  \path [line] (owns) -- (studio);
  \path [arrow] (runs) -- (studio);
  \path [arrow] (runs) -- (president);

\end{tikzpicture}
\end{figure}
Every studio has 0 or 1 studio.\\
Why is a president not a weak entity?\\
I don't know?\\

\end{document}
