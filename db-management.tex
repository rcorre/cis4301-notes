\documentclass[12pt]{article}
\usepackage{graphicx}
\usepackage{float}
\usepackage{amsmath}
\usepackage{sidecap}
\usepackage{fullpage}
\usepackage{hyperref}
\usepackage{listings}
\usepackage{latexsym}
\usepackage{color}
\usepackage{tikz}
\usetikzlibrary{shapes,arrows, matrix, positioning, fit}

% Java code with lstlisting
\definecolor{dkgreen}{rgb}{0,0.6,0}
\definecolor{gray}{rgb}{0.5,0.5,0.5}
\definecolor{mauve}{rgb}{0.58,0,0.82}

\lstset{frame=tb,
    language=Java,
    aboveskip=3mm,
    belowskip=3mm,
    showstringspaces=false,
    columns=flexible,
    basicstyle={\small\ttfamily},
    numbers=none,
    numberstyle=\tiny\color{gray},
    keywordstyle=\color{blue},
    commentstyle=\color{dkgreen},
    stringstyle=\color{mauve},
    breaklines=true,
    breakatwhitespace=true
    tabsize=3
}

\title{CIS4301 Notes: }
\author{Ryan Roden-Corrent}
\date{\today}

\begin{document}
\setlength\parindent{0pt}
% Tikz general settings
\tikzstyle{relation} = [diamond, draw, fill=blue!20, text width=4em,
  text badly centered, node distance=3cm, inner sep=0pt]
\tikzstyle{attribute} = [draw, ellipse, fill=red!20, node distance=2.5cm,
  minimum height=2em]
\tikzstyle{entity} = [rectangle, draw, fill=blue!20, text width=5em,
  text centered, minimum height=4em]
\tikzstyle{relation-weak} = [diamond, double, draw, fill=blue!20, text width=4em,
  text badly centered, node distance=3cm, inner sep=0pt]
\tikzstyle{entity-weak} = [rectangle, draw, double, fill=blue!20, text width=5em,
  text centered, minimum height=4em]
\tikzstyle{line} = [draw, -]
\tikzstyle{arrow} = [draw, -latex', thick]
\tikzstyle{arrow-round} = [draw, -), thick]
\maketitle

\section{Database Modifications}
\subsection{Insert}
\begin{lstlisting}[language=sql,caption=multiple value insertion]
  INSERT INTO Likes
  VALUES ('Sally', 'Bud'), ('Jim','Miller');  --comma separated tuples to enter
\end{lstlisting}
\subsubsection{Default Values}
\begin{lstlisting}[language=sql,caption=price defaults to 5 if not specified]
  ...,
  price Real DEFAULT 5,   --make sure price is a reasonable, non-NULL value
  ...,
\end{lstlisting}
\begin{lstlisting}[language=sql,caption=another default example]
  CREATE TABLE Drinkers (
    name CHAR(30) PRIMARY KEY,
    addr CHAR(50)
      DEFAULT '123 Sesame St.',
    phone CHAR(16)
  );
\end{lstlisting}
\subsubsection{Subqueries in insertion}
\begin{lstlisting}[language=sql,caption=insertion via subquery]
  INSERT INTO PotBuddies
  (
    SELECT d2.drinker
    FROM Frequents d1, Frequents d2
    WHERE d1.drinker = 'Sally AND
    d2.drinker <> 'Sally' AND
    d1.bar = d2.bar
  );
\end{lstlisting}
Find all the drinkers at the bars Sally frequents and insert them into
PotBuddies (Potential Buddies, what did you think it stands for?)\\
\begin{tabular}{|c|c|}
  \hline
  d1.bar & d2.bar\\
  \hline
  'Sally' & NOT 'Sally'\\
  'Sally' & NOT 'Sally'\\
  \hline
\end{tabular}

\subsection{Deletion}
\begin{lstlisting}[language=sql, caption=Sally no longer likes Bud]
  DELETE FROM Likes
  WHERE drinker = 'Sally' AND
    beer = 'Bud';
\end{lstlisting}
Delete all rows where the drinker is 'Sally' and the beer is 'Bud'
\begin{lstlisting}[language=sql,caption=clear out entire table]
  DELETE FROM Likes;  -- no WHERE clause needed
\end{lstlisting}

\begin{lstlisting}[language=sql,caption=delete with subquery]
  DELETE FROM Beers b
  WHERE EXISTS (  --check if another beer is made by the same manufacturer
    SELECT name FROM Beers  --implicit join of Beers with itself
    WHERE manf = b.manf AND
      name <> b.name
  );
\end{lstlisting}
Delete all beers where there is another beer by the same manufacturer.\\
\begin{tabular}{|c|c|c}
  \hline
  name & manf&\\
  \hline
  Bud & Budweiser & mark as dirty\\
  BudLite & Budweiser & mark as dirty\\
  \hline
\end{tabular}\\
Delete is a \textbf{mark-and-sweep} process: first mark items for deletion, then
delete all marked items. (If items were deleted immediately, it could disrupt
the condition for deleting other items during the same deletion process).

\subsection{Updates}
\begin{lstlisting}[language=sql,caption=UPDATE template]
  UPDATE <relation>
  SET <list of attribute assignments>
  WHERE <condition on tuples>;
\end{lstlisting}

\begin{lstlisting}[language=sql,caption=Change Fred's Phone number]
  UPDATE Drinkers
  SET phone = '555-1212'
  WHERE name = 'Fred';
\end{lstlisting}

\begin{lstlisting}[language=sql,caption=set maximum price on beers]
  UPDATE Sells
  SET price = 4.00
  WHERE price > 4.00;
\end{lstlisting}

\begin{lstlisting}[language=sql,caption=add tax to price]
  UPDATE Sells
  SET price = 1.05 * price  --value can be result of a computation on attributes
  WHERE price > 4.00;
\end{lstlisting}

\section{Constraints}
\begin{description}
    \item[constraint] relations enforced by DBMS
    \item[trigger] only executed when a condition occurs
\end{description}
\begin{description}
    \item[Keys] must be unique and non-null
    \item[Foreign-keys] referential integrity
    \item[value-based] constrain value of attribute
    \item[tuple-based] relationships between components
    \item[assertions] boolean expression
\end{description}
\subsection{Keys}
\subsubsection{Single Attribute Keys}
\begin{lstlisting}[language=sql,caption=ensure names are unique]
  CREATE TABLE Beers (
    name CHAR(20) UNIQUE,   --note: name can still be NULL!
    manf CHAR(20)
  );
\end{lstlisting}

\subsubsection{Multi Attribute Keys}
\begin{lstlisting}[language=sql,caption=tuple as a primary key]
  CREATE TABLE Sells (
  bar CHAR(20),
  beer VARCHAR(20),
  price REAL,
  PRIMARY KEY (bar,beer));
\end{lstlisting}

\subsubsection{Foreign Keys}
Indicate that a key REFERENCES another relation and is used as a key.\\
Referenced attributes must be declared PRIMARY KEY or UNIQUE.
\begin{lstlisting}[language=sql,caption=Foreign key]
  CREATE TABLE Beers (
    name CHAR(20) PRIMARY KEY,
    manf CHAR(20));
  CREATE TABLE Sells (
    bar CHAR(20),
    beer CHAR(20) REFERENCES Beers(name),
    price REAL);
\end{lstlisting}
\subsubsection{Enforcing Foreign Key Constraints}
If there is a foreign key constraint from R to S, two violations are possible:
\begin{itemize}
  \item An insert/update to R introduces values not found in S
  \item A delete/update to S causes some tuples of R to "dangle"
\end{itemize}
Actions Taken:
\begin{description}
  \item[Default] Reject the modification
  \item[Cascade] {
      Make the same changes in Sells
      \begin{itemize}
        \item Deleted beer: delete Sells tuple
        \item Updated beer: change value in Sells
      \end{itemize}
    }
  \item[Set NULL] Change the beer to NULL
\end{description}

Example Cascade:\\
Delete Bud tuple from Beers $\rightarrow$ delete all tuples from sells that have
beer = 'bud'
Example Cascade:\\
Delete Bud tuple from Beers $\rightarrow$ change tuples from sells that have
beer = 'bud' to have beer = NULL

\begin{lstlisting}[language=sql,caption=setting policy]
  CREATE TABLE Sells (
    bar CHAR(20),
    beer CHAR(20) REFERENCES Beers(name),
    price REAL,
    FOREIGN KEY(beer)
      REFERENCES Beers(name)
      ON DELETE SET NULL
      ON UPDATE CASCADE);
\end{lstlisting}

\subsubsection{Attribute level checks}
Add CHECK(<condition>) to the declaration for the attribute.
The condition may use the name of the attribute but
\textcolor{red}{any other relation or attribute name must be in a subquery.}

\begin{lstlisting}[language=sql,caption=attribute check]
  CREATE TABLE Sells (
    bar CHAR(20),
    beer CHAR(20) CHECK (beer IN  --make sure beer exists in Beers relation
      (SELECT name FROM Beers)),  --implementation of FOREIGN KEY with a CHECK
    price REAL CHECK ( price <= 5.00)   --make sure price is >= 5.00
  );
\end{lstlisting}
Select subqueries inside check (like above) are \textbf{not} supported in Postgres
\begin{lstlisting}[language=sql,caption=tuple-based check]
  CREATE TABLE Sells (
    bar CHAR(20),
    beer CHAR(20),
    price REAL,
    CHECK (bar = 'Joe''s' OR    --allow Joe to sell beer at > 5.00
      price <= 5.00));
  );
\end{lstlisting}
\begin{lstlisting}[language=sql,caption=operation in check]
  CREATE TABLE Sells (
    bar CHAR(20),
    beer CHAR(20),
    price REAL,
    tax REAL DEFAULT .05,
    CHECK (price * tax <= 5.00));
\end{lstlisting}

\subsubsection{Assertions}
Can be messy, try to avoid. Just know that they are schema-level constraints
\begin{lstlisting}[language=sql,caption=assertion]
  CREATE ASSERTION NoRipoffBars CHECK (
    NOT EXISTS (  --no bar may charge an average of more than 5.00
      SELECT bar FROM Sells
      GROUP BY bar
      HAVING 5.00 < AVG(price)
  ));
\end{lstlisting}

\subsection{Triggers}
Motivation:
\begin{itemize}
  \item Assertions are powerful, but cannot turn them off.
  \item attribute-level checks are controllable but not powerful
  \item Triggers let the user decide when they should be checked.
\end{itemize}

Also known as \textbf{ECA (event-condition-action)} rule.
\begin{description}
  \item[Event] type of db modification (e.g. insert on sells)
  \item[Condition] SQL boolean expression
  \item[Action] SQL statement
\end{description}

Example: After every insertion on sells, check the Beers table to see if the
beer you are trying to insert exists. If it doesn't, insert that name into the
Beers table.
\begin{lstlisting}[language=sql,caption=Trigger definition]
  CREATE TRIGGER BeerTrig
  AFTER INSERT ON Sells         --the event
  REFERENCING NEW ROW AS NewTuple
  FOR EACH ROM
  WHEN (NewTuple.beer NOT IN    --the condition
    (SELECT name FROM Beers))
    INSERT INTO Beers(name)     --the action
      VALUES(NewTuple.beer);
\end{lstlisting}

\subsubsection{Options}
\begin{itemize}
  \item CREATE TRIGGER $<name>$
  \item {
    CREATE OR REPLACE TRIGGER $<name>$
  }
\end{itemize}
Triggers:
\begin{itemize}
  \item BEFORE
  \item AFTER
  \item INSTEAD OF - used for Views
\end{itemize}



\section{Transactions, Views, and Indexes}
A relation defined in terms of stored tables (\textbf{base tables}) and other
views
\begin{lstlisting}[language=sql, caption=virtual view creation]
  CREATE VIEW beer_view AS
    SELECT beer, price
    FROM Products
    WHERE type='beer';
\end{lstlisting}
\begin{description}
  \item[virtual] not stored in the database, just a query for constructing a
    relation \textbf{(default)}
  \item[materialized] actually constructed and stored
\end{description}
\begin{lstlisting}[language=sql, caption=virtual view usage]
  SELECT Count(*)     --get number of beers that start with 'Bud'
    FROM beer_view
    WHERE name LIKE 'Bud\%'
    GROUP BY name;
\end{lstlisting}
Views allow people to work with a part of the database without giving them full
access.

\subsection{Materialized Views}
\begin{lstlisting}[language=sql, caption=materialized view]
  CREATE MATERIALIZED VIEW ...
\end{lstlisting}
If a change is made to a base table that a virtual view is based on, the view is
marked as dirty and must be reconstructed on the next access.
A \textbf{materialized view} checks whether the update is relevant and performs
that update on itself. For example, adding one element to the base table adds
that element to the view.
\subsection{Triggers on Views}
\begin{itemize}
  \item generally cannot modify a virtual view because it doesn't exist
  \item INSTEAD OF lets us interpret view modifications in a sensible way
\end{itemize}
\subsubsection{Example}
View Synergy has \textcolor{red}{(drinker, beer, bar)}

\begin{lstlisting}[language=sql, caption=view trigger]
  CREATE TRIGGER ViewTrig   --simulate insertion into Synergy
  INSTEAD OF INSERT ON Synergy
  REFERENCING NEW ROW AS n  --n: row you were trying to insert
  FOR EACH ROW
  BEGIN
    INSERT INTO LIKES VALUES(n.drinker, n.beer);
    INSERT INTO SELLS(bar,beer) VALUES(n.bar, n.beer);
    INSERT INTO FREQUENTS VALUES(n.drinker, n.bar);
  END;
\end{lstlisting}
\subsection{Data Warehouse Example}
\begin{itemize}
  \item WalMart stores every sale at every store
  \item overnight, sales for the date are used to update a
    \textcolor{red}{data warehouse}, which is a materialized view of the sales
  \item the warehouse is used by analysts to predict trends
\end{itemize}
Might use a view to represent global scales in a signle table independent of
local currencies.

\subsection{Indexes}
Indexes are data structures used to speed access to tuples of a relation given
one or more attributes.
It could be a hash table, but in a DMBS is is always a balanced search tree
with giant nodes (a full disk page) called a \textbf{B-tree}
\subsubsection{Example}
\begin{lstlisting}[language=sql, caption=a long computation]
  SELECT county Crime
  FROM Crimes
  WHERE State='FL';   --how many crimes were committed in florida
\end{lstlisting}
Creating an index on some columns of the table (for example, \textbf{state}).
This would create a hash table where each key (a state) points to every
instance of that state in the main table.

\subsubsection{Another Example}
Note: use $\backslash$timing to measure speed in sql.
\begin{lstlisting}[language=sql, caption=a long computation]
  select count(*) from olympics where country = 'United States';
\end{lstlisting}
\begin{lstlisting}[language=sql, caption=create index to speed count]
  CREATE INDEX country_idx ON olympics (country);
\end{lstlisting}

\section{SQL Functions}
use $\backslash$s to list functions in your database.
\subsection{User-defined Functions}
\begin{lstlisting}[language=sql]
SELECT a, CASE  WHEN a=1 THEN 'one'
                WHEN a=2 THEN 'two'
                ELSE 'don''t care'
          END
\end{lstlisting}


\begin{lstlisting}[language=sql, caption=example function]
CREATE OR REPLACE FUNCTION add_em(x integer, y integer) RETURNS integer AS
$$
SELECT x + y;
$$ LANGUAGE SQL;
\end{lstlisting}

\begin{lstlisting}[language=sql, caption=alternate argument reference style]
CREATE OR REPLACE FUNCTION add_em(x integer, y integer) RETURNS integer AS
$$
SELECT $1 + $2;   --bash-style argument references, same result as above
$$ LANGUAGE SQL;
\end{lstlisting}

\textbf{Languages:}
\begin{itemize}
  \item pl/python
  \item pl/perl
  \item pl/pgsql
  \item pl/tcl
  \item c
\end{itemize}

\begin{description}
  \item[trusted languages] lua, perl, ect.
  \item[untrusted languages] java, python, ect.
\end{description}
Untrusted languages could potentially execute damaging commands with superuser
priveleges.

\begin{lstlisting}[language=sql, caption=another function]
--good practice to prefix function with owner name
CREATE OR REPLACE FUNCTION cgrant_dont_carem(x integer) RETURNS void AS
$$
  DELETE FROM test
    WHERE a < x;    --delete every entry in test where a is less than the arg
$$ LANGUAGE SQL;
\end{lstlisting}

\begin{lstlisting}[language=sql, caption=calling a function]
  select cgrant_dont_care(3);
\end{lstlisting}

\begin{lstlisting}[language=sql, caption=function with a range]
--note: this is an error if the function above was previously defined
--cannot replace function if changing signature
CREATE OR REPLACE FUNCTION cgrant_dont_carem(start integer, end integer) RETURNS
void AS
$$
  DELETE FROM test
    WHERE a > start AND a < end;
$$ LANGUAGE SQL;
\end{lstlisting}

\begin{lstlisting}[language=sql, caption=function with a range (properly
  defined)]
--this will work
CREATE FUNCTION cgrant_dont_carem(start integer, end integer) RETURNS
void AS
$$
  DELETE FROM test
    WHERE a > start AND a < end;
$$ LANGUAGE SQL;
\end{lstlisting}

\begin{lstlisting}[language=sql, caption=delete function]
  DROP FUNCTION cgrant_dont_care(integer);
\end{lstlisting}

\subsection{SQL Binary Storage}
Many DBs have BLOB (Binary Large Object)\\
Postgres has bytea (byte array)
\section{Assignment 3 Review}
\subsection{Olympic Queries}
\subsubsection{Olympic Queries}
\end{document}
