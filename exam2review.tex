\documentclass[12pt]{article}
\usepackage{graphicx}
\usepackage{float}
\usepackage{amsmath}
\usepackage{sidecap}
\usepackage{fullpage}
\usepackage{hyperref}
\usepackage{listings}
\usepackage{latexsym}
\usepackage{color}
\usepackage{tikz}
\usetikzlibrary{shapes,arrows, matrix, positioning, fit}

% Java code with lstlisting
\definecolor{dkgreen}{rgb}{0,0.6,0}
\definecolor{gray}{rgb}{0.5,0.5,0.5}
\definecolor{mauve}{rgb}{0.58,0,0.82}

\lstset{frame=tb,
    language=Java,
    aboveskip=3mm,
    belowskip=3mm,
    showstringspaces=false,
    columns=flexible,
    basicstyle={\small\ttfamily},
    numbers=none,
    numberstyle=\tiny\color{gray},
    keywordstyle=\color{blue},
    commentstyle=\color{dkgreen},
    stringstyle=\color{mauve},
    breaklines=true,
    breakatwhitespace=true
    tabsize=3
}

\title{CIS4301 Notes: Exam 2 Review}
\author{Ryan Roden-Corrent}
\date{\today}

\begin{document}
\setlength\parindent{0pt}
% Tikz general settings
\tikzstyle{relation} = [diamond, draw, fill=blue!20, text width=4em,
  text badly centered, node distance=3cm, inner sep=0pt]
\tikzstyle{attribute} = [draw, ellipse, fill=red!20, node distance=2.5cm,
  minimum height=2em]
\tikzstyle{entity} = [rectangle, draw, fill=blue!20, text width=5em,
  text centered, minimum height=4em]
\tikzstyle{relation-weak} = [diamond, double, draw, fill=blue!20, text width=4em,
  text badly centered, node distance=3cm, inner sep=0pt]
\tikzstyle{entity-weak} = [rectangle, draw, double, fill=blue!20, text width=5em,
  text centered, minimum height=4em]
\tikzstyle{line} = [draw, -]
\tikzstyle{arrow} = [draw, -latex', thick]
\tikzstyle{arrow-round} = [draw, -), thick]
\maketitle

\section{Things to Review}
\begin{itemize}
  \item {Normal Form: \textbf{1NF, 2NF, 3NF}}
  \item {Functional Dependencies
      \begin{itemize}
        \item {6 Rules, \textbf{especially transitive}}
      \end{itemize}
    }
  \item {More queries
      \begin{itemize}
        \item {GROUP BY}
        \item {NULLs}
        \item {Aggregates (and their behavior with regards to NULL)}
      \end{itemize}
    }
  \item {Indices: When is it a good idea to use indexes?
      \begin{itemize}
        \item {For example, given a set of queries, use indices to speed it up}
      \end{itemize}
    }
  \item {Views: virtual vs materialized (how are they updated?)}
  \item {Foreign Keys, Check Constraints}
  \item {Constraints
      \begin{itemize}
        \item {Table Level}
        \item {Tuple Level}
      \end{itemize}
    }
  \item {Triggers: After vs Before:
    \begin{itemize}
      \item {Before: Allows trigger to throw error state and stop further
          modification}
      \item {After: Ok if not too much to undo}
    \end{itemize}
  }
\end{itemize}

\section{Chapter Breakdown}
Ch.3, Ch.5-9 are on the test.\\
\subsection{Stuff you don't need to study:}
\fbox{you \textcolor{red}{\textbf{DON'T}} need to know the following:}
\begin{description}
  \item[Ch.3] Closure FD, set cover,  BCNF, 4NF, 5NF
  \item[Ch.5]
  \item[Ch.6] 6.6.4-6.6.6
  \item[Ch.7] 7.4
  \item[Ch.8] 8.4
  \item[Ch.9] 9.1 - 9.3, 9.5-9.7
  \item[9.4] probably don't need to know stored procedures for quiz, but good to
    know
\end{description}

\section{Indexes (Indices?)}
An index can be thought of as a \textbf{hashtable} or \textbf{b-tree}.
\begin{description}
  \item[hashtable] {quick lookups, point queries
      \begin{itemize}
        \item {Point Query: Avoid scanning large set, go directly to record}
      \end{itemize}
    }
  \item[b-tree] {Range Queries
      \begin{itemize}
        \item {very flat, holds alot of information}
        \item {useful for e.g. get all items with $attribute > some\_value$}
        \item {such a query would not work well with hashtable (repeated hash
            value computation)}
      \end{itemize}
    }
\end{description}
\subsection{Why not always use Indices?}
Some databases kinda do (e.g. MongoDB). However,
If you index everything, you are creating alot of duplication. Creating a
specialized data structure for every column \textbf{consumes excessive space},
and \textbf{updates can take a long time}.
\subsection{What if something is updated in database but not index}
In general, must check if cached value is marked as "dirty".
\section{Window Functions}
Look them up on postgres docs, especially \textbf{rank()}.\\
\begin{lstlisting}[language=sql,caption=rank example]
  SELECT X, avg(salary) OVER (PARTITION BY X)
  FROM G
  GROUP BY X;
\end{lstlisting}
\end{document}
