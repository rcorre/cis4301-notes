\documentclass[12pt]{article}
\usepackage{graphicx}
\usepackage{float}
\usepackage{amsmath}
\usepackage{sidecap}
\usepackage{fullpage}
\usepackage{hyperref}
\usepackage{listings}
\usepackage{latexsym}
\usepackage{color}
\usepackage{tikz}
\usetikzlibrary{shapes,arrows, matrix, positioning, fit}

% Java code with lstlisting
\definecolor{dkgreen}{rgb}{0,0.6,0}
\definecolor{gray}{rgb}{0.5,0.5,0.5}
\definecolor{mauve}{rgb}{0.58,0,0.82}

\lstset{frame=tb,
    language=Java,
    aboveskip=3mm,
    belowskip=3mm,
    showstringspaces=false,
    columns=flexible,
    basicstyle={\small\ttfamily},
    numbers=none,
    numberstyle=\tiny\color{gray},
    keywordstyle=\color{blue},
    commentstyle=\color{dkgreen},
    stringstyle=\color{mauve},
    breaklines=true,
    breakatwhitespace=true
    tabsize=3
}

\title{CIS4301 Guest Speaker: App Prototyping}
\author{Ryan Roden-Corrent}
\date{\today}

\begin{document}
\setlength\parindent{0pt}
% Tikz general settings
\tikzstyle{relation} = [diamond, draw, fill=blue!20, text width=4em,
  text badly centered, node distance=3cm, inner sep=0pt]
\tikzstyle{attribute} = [draw, ellipse, fill=red!20, node distance=2.5cm,
  minimum height=2em]
\tikzstyle{entity} = [rectangle, draw, fill=blue!20, text width=5em,
  text centered, minimum height=4em]
\tikzstyle{relation-weak} = [diamond, double, draw, fill=blue!20, text width=4em,
  text badly centered, node distance=3cm, inner sep=0pt]
\tikzstyle{entity-weak} = [rectangle, draw, double, fill=blue!20, text width=5em,
  text centered, minimum height=4em]
\tikzstyle{line} = [draw, -]
\tikzstyle{arrow} = [draw, -latex', thick]
\tikzstyle{arrow-round} = [draw, -), thick]
\maketitle

\section{Users Goals Metrics}
\begin{itemize}
  \item Who is the user? What are their goals?
  \item Come up with a "persona" to represent the people who will
    be using your app.
  \item Have general "metrics of success"
  \item One-size-fits-all software doesn't really appeal to anyone in particular
\end{itemize}

\subsection{example}
\begin{description}
  \item[Developer]
  \item[IT Manager]
  \item[CIO]
\end{description}

\subsection{5 person useability studies}
Often just testing with 5 people can give great insight into what works and
what doesn't in terms of the user interface. Track the errors they make
(thinking they were clicking on something when they really weren't) and tweak
the software. Need to make sure you have the appropriate testers for the
software.

\section{Strategy}
\begin{description}
  \item [Exit Strategy] What will you do if it takes off?
    Is it a full time project or just a project done out of curiosity?
  \item [Defense] What makes your app different from what already exists?
  \item [Funding source] dictates how you can build features
\end{description}

\subsection{Funding}
\begin{description}
  \item [bootstrap] Your own money, you dicate how to build the software
  \item [grant] grants tend not to care as much how it is build, just whether
    it achieves the end goal
\end{description}
With grants, need to know when to stop. Don't undervalue your time.

\section{Flow}
\begin{description}
  \item[Flow Diagram] Create before writing code, can save time in the long run.
  \item[Wire Frame] Representation of user interfaces. Each wireframe should
    have an associated persona and goal. Appearance not important, very high
    level
  \item[Composite] Mock-up, generally looks better than wireframe
  \item[Paper Prototype] Walk someone through using a paper version of the app
  \item[Define MVP] Minimally Viable Product: what is the minimum required for
    app to "work"
  \item[Prioritizing Features] what are the most important features? What are
    blocking features? In what order should features be completed so everyone
    can work in parallel?
  \item[Tickets] use goal tracker (e.g. Trello) and issue tracker (e.g. github
    issues.)
  \item[Sprints] define a time period for work, set backlog during planning
    meeting, typically have once-a-day $\approx$ 15min "standup" meeting
  \item[Quality Assurance] mutliple environments (work, staging/integration,
    production). Test in an environment exactly like the "real-world". Run test
    suites.
  \item[Release]
  \item[Retrospective]
\end{description}

\section{Common Trouble}
\begin{description}
  \item[Scope creep] Keep adding new features, which delays release and bloats
    product
  \item[Nebulous Hierarchy] can't have a "flat" team. Assign specialties or
    lead roles to members.
  \item[Too many stakeholders] Who "says yes" for a particular kind of decision?
  \item[Automating too soon] don't over-engineer too soon. Just make it happen
    as quick as possible
\end{description}

\section{Stack}
Make sure you are using the appropriate technology for the project

\end{document}
