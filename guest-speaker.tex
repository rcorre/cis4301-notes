\documentclass[12pt]{article}
\usepackage{graphicx}
\usepackage{float}
\usepackage{amsmath}
\usepackage{sidecap}
\usepackage{fullpage}
\usepackage{hyperref}
\usepackage{listings}
\usepackage{latexsym}
\usepackage{color}
\usepackage{tikz}
\usetikzlibrary{shapes,arrows, matrix, positioning, fit}

\title{CIS4301 Guest Speakers}
\author{
  Speakers: Dave Stanton (@gotoplanb),
  Sean Goldberg(sean@cise.ufl.edu)
  Pierre St. Juste
  Nick Antonelli (Grooveshark)\\
  Notes: Ryan Roden-Corrent
}
\date{\today}

\begin{document}
\setlength\parindent{0pt}
% Tikz general settings
\tikzstyle{relation} = [diamond, draw, fill=blue!20, text width=4em,
  text badly centered, node distance=3cm, inner sep=0pt]
\tikzstyle{attribute} = [draw, ellipse, fill=red!20, node distance=2.5cm,
  minimum height=2em]
\tikzstyle{entity} = [rectangle, draw, fill=blue!20, text width=5em,
  text centered, minimum height=4em]
\tikzstyle{relation-weak} = [diamond, double, draw, fill=blue!20, text width=4em,
  text badly centered, node distance=3cm, inner sep=0pt]
\tikzstyle{entity-weak} = [rectangle, draw, double, fill=blue!20, text width=5em,
  text centered, minimum height=4em]
\tikzstyle{line} = [draw, -]
\tikzstyle{arrow} = [draw, -latex', thick]
\tikzstyle{arrow-round} = [draw, -), thick]
\maketitle

\section{App Development (Dave Stanton)}
\subsection{Users Goals Metrics}
\begin{itemize}
  \item Who is the user? What are their goals?
  \item Come up with a "persona" to represent the people who will
    be using your app.
  \item Have general "metrics of success"
  \item One-size-fits-all software doesn't really appeal to anyone in particular
\end{itemize}

\subsubsection{example}
\begin{description}
  \item[Developer]
  \item[IT Manager]
  \item[CIO]
\end{description}

\subsubsection{5 person useability studies}
Often just testing with 5 people can give great insight into what works and
what doesn't in terms of the user interface. Track the errors they make
(thinking they were clicking on something when they really weren't) and tweak
the software. Need to make sure you have the appropriate testers for the
software.

\subsection{Strategy}
\begin{description}
  \item [Exit Strategy] What will you do if it takes off?
    Is it a full time project or just a project done out of curiosity?
  \item [Defense] What makes your app different from what already exists?
  \item [Funding source] dictates how you can build features
\end{description}

\subsubsection{Funding}
\begin{description}
  \item [bootstrap] Your own money, you dicate how to build the software
  \item [grant] grants tend not to care as much how it is build, just whether
    it achieves the end goal
\end{description}
With grants, need to know when to stop. Don't undervalue your time.

\subsection{Flow}
\begin{description}
  \item[Flow Diagram] Create before writing code, can save time in the long run.
  \item[Wire Frame] Representation of user interfaces. Each wireframe should
    have an associated persona and goal. Appearance not important, very high
    level
  \item[Composite] Mock-up, generally looks better than wireframe
  \item[Paper Prototype] Walk someone through using a paper version of the app
  \item[Define MVP] Minimally Viable Product: what is the minimum required for
    app to "work"
  \item[Prioritizing Features] what are the most important features? What are
    blocking features? In what order should features be completed so everyone
    can work in parallel?
  \item[Tickets] use goal tracker (e.g. Trello) and issue tracker (e.g. github
    issues.)
  \item[Sprints] define a time period for work, set backlog during planning
    meeting, typically have once-a-day $\approx$ 15min "standup" meeting
  \item[Quality Assurance] mutliple environments (work, staging/integration,
    production). Test in an environment exactly like the "real-world". Run test
    suites.
  \item[Release]
  \item[Retrospective]
\end{description}

\subsection{Common Trouble}
\begin{description}
  \item[Scope creep] Keep adding new features, which delays release and bloats
    product
  \item[Nebulous Hierarchy] can't have a "flat" team. Assign specialties or
    lead roles to members.
  \item[Too many stakeholders] Who "says yes" for a particular kind of decision?
  \item[Automating too soon] don't over-engineer too soon. Just make it happen
    as quick as possible
\end{description}

\subsection{Stack}
Make sure you are using the appropriate technology for the project

\section{Crowd Sourcing (Sean Goldberg)}
\subsection{Big Data}
\begin{description}
  \item[Volume] Existing data waiting to process
  \item[Velocity] Data in motion (Streaming)
  \item[Variety] structured, unstructured, text, multimedia
  \item[Veracity] data in doubt (inconsistency, incompleteness, ambiguity,
    latency)
\end{description}
Want to organize this data. Example: Wikipedia info box.

\subsection{Machine Learning}
training set $\rightarrow$ learning algorithm $\rightarrow$ heuristic

\subsubsection{Examples of use:}
\begin{itemize}
  \item Self-driving car
  \item Machine translation (like Babel Fish from Hitchhiker's Guide)
  \item Amazon reccommendations
  \item Apple's Siri (natural language question $\rightarrow$ database query)
  \item license plate processing
  \item Watson (jeopardy playing robot)
\end{itemize}

\subsubsection{Current Machine Learning Limitations}
\begin{itemize}
  \item Natural Language Processing still limited
\end{itemize}

\subsection{Crowd Sourcing}
Human processing power can make up for areas that machines lack. Try to combine
human and machine computation to complete a goal.\\
Bringing in a single human is not efficient. Need an effective way to source
the collective knowledge of many humans.\\
\subsubsection{Amazon Mechanical Turk}
Provides a marketplace for sourcing human intelligence.
\begin{itemize}
  \item Make an open call for jobs
  \item People sign up for jobs
  \item People complete task with human intelligence
  \item People get payed
  \item Allows parallelization of tasks
\end{itemize}
Example use: Collecting Training data for an intelligent system. If you are
training a system to identify emotions by looking at a picture, you need people
to look at the pictures and classify them so the machine can learn from them.

\subsubsection{TurKit}
A programming toolkit that integrates human intelligence as subroutines in your
code.

\subsubsection{Soylent}
A MS Word plugin using people to perform word processing tasks that are
difficult fr a computer (shortening, proofreading, macro)

\subsubsection{CrowdDb}
Use human intelligence to correct and optimize a database in ways that are
difficult for a machine.

\subsubsection{DbPedia}
Make wikipedia queryable like a database by using human intelligence.

\subsection{CASTLE (Crowd Assisted System for Text Labeling and Extraction)}
Converting unstructured text into a structured form (database) that can be
queried.

Combine probablilistic database with crowd knowledge.\\
Run a machine learning algorithm over text to tag tokens. The algorithm
generates a set of possible tags, each with a given correctness probablility.
Identify uncertain tags and source them to humans by automatically posting HITs.

\section{
Distributed Databases and their applications: From DHTs to Memcache to Twister
}
By Pierre St. Juste\\
\begin{itemize}
  \item Distributed Hash Tables (Memcache, Dynamo)
  \item Distributed Datastores (Bigtable, MongoDB)
  \item Real Applications (Bitweav, Twister, Litter)
\end{itemize}

\subsection{Centralized Data Storage}
\fbox{Presentation tier} $\leftrightarrow$
\fbox{Application Tier} $\leftrightarrow$
\fbox{Data Tier}\\
Infeasible for large amounts of data with frequent access.
\subsection{Distributed Data Storage}
Increase performance through parallelism
\subsubsection{Amazon Dynamo}
\begin{itemize}
  \item NoSQL key-value
  \item Accesible via HTTP REST API
  \item Data Auto-partitioned
\end{itemize}
Different macines responsible for storing different ranges of keys.
Want to ensure key distribution is random to achieve good load balancing (need
a strong hash function). Good load balancing can avoid performance bottleneck
and network bottleneck.

\subsubsection{Questions about key-value stores}
\begin{description}
  \item[handling replication?] Generate multiple key-val pairs with different
    keys but same value
  \item[load balancing?] Strong (sha1) hash of key
  \item[map table to KV store?] map each sell to a key (e.g. key =
    tablename\_primarykey\_column)
\end{description}

\subsection{Memcache}
Designed to work alongside a database.
\begin{itemize}
  \item all information stored on disk
  \item key-value pair has no data-type restrictions
  \item read-only cache for information stored in DB
  \item Memcache server failure is not critical
\end{itemize}
A number of memcache servers store information that is requested from DB. A
request first checks the Memcache, and only has to check the DB if the
requested value isn't present in the Memcache.

\subsubsection{Limitations of key-value stores}
\begin{itemize}
  \item SQL like queries on DHT can be ineficcient
\end{itemize}

\subsection{Google Bigtable}
\begin{itemize}
  \item Read a single row or range
  \item can request timestamp range
  \item implemented as multi-map
  \item sections of table stored as "tablets" across servers
  \item meta data servers track location of tablets
  \item key-value lookup performed to retrieve table contents
  \item \textbf{More efficient SQL-like lookups}
\end{itemize}

\subsection{MongoDB}
\begin{itemize}
  \item Schema-free
  \item Document oriented
  \item Scaling and load-balancing through sharding
  \item Store arbitrary objects directly in database
\end{itemize}

\subsubsection{Sharding}
Break a key space evenly across servers.

\subsection{Distributed Microblogging}
\subsubsection{Twister}
\begin{itemize}
  \item P2P anonymous microblogging service
  \item {
    use Bitcoin for user registration
    \begin{itemize}
      \item everyone is a witness to other's transactions
    \end{itemize}
  }
  \item {
    use Bittorrent DHT for data storage
    \begin{itemize}
      \item download torrent file to get keys
      \item values are IP addresses of people who have needed data
      \item in this case, can be used to store tweets in a decentralized manner
    \end{itemize}
  }
\end{itemize}
\subsubsection{Litter}
\begin{itemize}
  \item each node runs a simple python webserver
  \item each node uses SQLite database
  \item Updates are pushed through IP Multicast
  \item \url{https://github.com/ptony82/litter}
\end{itemize}

\section{Grooveshark Architecture, Managing DB at Scale, NoSQL}
\begin{itemize}
  \item multi-tiered stack
  \item CACHE EVERYTHING
  \item ansynchronous code is your friend
  \item different workloads $\rightarrow$ different DBMSes (don't make sql do
    everything)
\end{itemize}
\subsection{The Website}
\begin{itemize}
  \item several front-end servers with load balancer
  \item requests are routed to a random front-end server
  \item only load some of the data synchronously
  \item load most of the data asynchronously
  \item get the user on the site as fast as possible!
\end{itemize}
\subsection{Gearman / Memcached}
\begin{itemize}
  \item Most frontend GS calls are asynchronous
  \item gearman/memcached serve as buffer to data stores and a common
    'language' for other services
  \item gearman performs actions at behest of user
\end{itemize}
\subsubsection{Gearman}
\begin{itemize}
  \item Asynchronous task manager
  \item job submitted to a queue
  \item worker processes listen on queues, process jobs as they come in
  \item stateless, workers just know what queues to listen on and what to do
    upon job arrival
\end{itemize}
\subsubsection{Memcached}
\begin{itemize}
  \item distributed hash table
  \item key-value store
  \item each memcached process is a bucket
  \item all library calls go to memcached first
  \item eliminate unnecessary calls to data stores, reduce cost of data
    retrieval
\end{itemize}
\subsection{Data Stores}
\begin{itemize}
  \item highly heterogenous environment (MariaDB, Percona, MongoDB ...)
  \item not all data stores are made the same
  \item different R/W tolerance due to differing data storage
  \item each DBMS excels in certain areas
\end{itemize}
\begin{itemize}
  \item only MySQL, Postgres, and Hive (kinda) use SQL
  \item do not maintain same guarantees as SQL (could be good thing)
  \item SQL: good for transactional, ACID complient storage, can be unwieldy
  \item NoSQL: no ACID guarantees but is more flexible (good for e.g. comment
    system)
  \item CAP Theorem
\end{itemize}
Commonalities among data stores:\\
\begin{itemize}
  \item caching
  \item indices!!!
  \item backups
\end{itemize}
\subsubsection{MySQL}
\begin{itemize}
  \item primary backend, go-to unless there is a use case for a different DB
  \item {
      use open source forks instead of Oracle MySQL
      \begin{itemize}
        \item MariaDB
        \item Percona
      \end{itemize}
    }
  \item top-down topology
  \item production master/slave
  \item used by front-end servers exclusively
\end{itemize}
\subsubsection{MongoDB}
\begin{itemize}
  \item distributed NoSQL system: node awareness, eventual consistency
  \item JSON-like syntax: type sensitive arrays and objects
  \item indices: more sensitive than SQL counterparts
\end{itemize}
\begin{lstlisting}
  use gs;
  db.myColl.find({x:5})
\end{lstlisting}
\subsubsection{Hive/HBase}
\begin{itemize}
  \item Part of apache Hadoop project
  \item Build on HDFS (Hadoop Filesystem)
  \item {
      Hive: ad-hoc data querying (data warehouse)
      \begin{itemize}
        \item transforms SQL to Map/Reduce job
      \end{itemize}
  }
  \item HBase: column store, easy aggregation, write-heavy workloads
\end{itemize}
\subsubsection{Redis}
\begin{itemize}
  \item in memory
  \item like memcached, but more fully featured
  \item sorted sets, lists
  \item mostly used for internal services
\end{itemize}
\end{document}
