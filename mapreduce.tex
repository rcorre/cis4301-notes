\documentclass[12pt]{article}
\usepackage{graphicx}
\usepackage{sverb}
\usepackage{float}
\usepackage{amsmath}
\usepackage{sidecap}
\usepackage{fullpage}
\usepackage{hyperref}
\usepackage{listings}
\usepackage{latexsym}
\usepackage{color}
\usepackage{tikz}
\usetikzlibrary{shapes,arrows, matrix, positioning, fit}

\lstset{
  backgroundcolor=\color{lightgray},
  extendedchars=true,
  basicstyle=\footnotesize\ttfamily,
  showstringspaces=false,
  showspaces=false,
  numbers=left,
  numberstyle=\footnotesize,
  numbersep=9pt,
  tabsize=2,
  breaklines=true,
  showtabs=false,
}

\title{CIS4301 Notes: Data Mining}
\author{Ryan Roden-Corrent}
\date{\today}
\begin{document}
\setlength\parindent{0pt}
% Tikz general settings
\tikzstyle{relation} = [diamond, draw, fill=blue!20, text width=4em,
  text badly centered, node distance=3cm, inner sep=0pt]
\tikzstyle{attribute} = [draw, ellipse, fill=red!20, node distance=2.5cm,
  minimum height=2em]
\tikzstyle{entity} = [rectangle, draw, fill=blue!20, text width=5em,
  text centered, minimum height=4em]
\tikzstyle{relation-weak} = [diamond, double, draw, fill=blue!20, text width=4em,
  text badly centered, node distance=3cm, inner sep=0pt]
\tikzstyle{entity-weak} = [rectangle, draw, double, fill=blue!20, text width=5em,
  text centered, minimum height=4em]
\tikzstyle{line} = [draw, -]
\tikzstyle{arrow} = [draw, -latex', thick]
\tikzstyle{arrow-round} = [draw, -), thick]
\maketitle
\section{MapReduce}
\href{http://cise.ufl.edu/class/cis4301sp14/slides/map_reduce.pdf}
{MapReduce Class Slides}
Popular in \textbf{functional} languages (Ruby, Lisp, Python).
\subsection{Functional Languages}
Functional languages treat functions as first class objects. Functions take
other functions as arguments and return them as results. Operators do not
mutate collections.
For example:
\begin{lstlisting}
  // isprof is a function that returns true if the argument is a professor
  // list is a list of people, some of which may be professors
  count(isprof, list);
\end{lstlisting}

\subsection{Common functions}
\begin{description}
  \item [map] apply a function to each element to create a new collection\\
    \lstinline{map (+3 [1,2,3,4,5] => [4,5,6,7,8]}
  \item [reduce] apply a function between each element and an accumulator to
    reduce a list to a single value
  \item [any] return true if a function returns true for $\ge$ 1 value
  \item [all]return true if a function returns true for all values
  \item [filter] keep only values where function returns true
  \item [fold] like reduce, but take an initial parameter for the accumulator
\end{description}

\subsection{Unix Philosophy}
Unix Philosophy: A tool should do one thing well. Complex processes should be
achieved by chaining several commands together.

\subsection{MapReduce}
Proposed by Google engineers. It includes:
\begin{itemize}
  \item {MapReduce programming model}
  \item {Google file system}
  \item {Job scheduling system}
\end{itemize}

\subsection {MapReduce programming model}
Generalize map and reduce funtions.
\begin{lstlisting}[caption=mapreduce example]
  map(String key, String value):
    // key: document name
    // value: document contents
    for each word w in value:
      EmitIntermediate(w, "l");
  reduce(String key, Iterator values):
    int result = 0;
    for each v in values:
      result += ParseInt(v);
      Emit(AsString(result));
\end{lstlisting}
This code runs on a large scale system. The map job is split up between
various machines. Once these results are obtained, they are passed on to
another set of machines for reduction. The process of assigning map results to
reduce machines is called the \textbf{shuffle} step.

\subsubsection{Another Example}
Suppose you need to count stars in a 1TB jpeg of the night sky. You could split
the image into sections, use map to count the stars in each section, and reduce
the section counts into a single sum.

\subsection{Hadoop}
Google's MapReduce is proprietary, Yahoo created an open source alternative
called Hadoop.

\end{document}
