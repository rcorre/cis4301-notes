\documentclass[12pt]{article}
\usepackage{graphicx}
\usepackage{float}
\usepackage{amsmath}
\usepackage{sidecap}
\usepackage{fullpage}
\usepackage{hyperref}
\usepackage{listings}
\usepackage{latexsym}
\usepackage{color}
\usepackage{tikz}
\usetikzlibrary{shapes,arrows, matrix, positioning, fit}

% Java code with lstlisting
\definecolor{dkgreen}{rgb}{0,0.6,0}
\definecolor{gray}{rgb}{0.5,0.5,0.5}
\definecolor{mauve}{rgb}{0.58,0,0.82}

\lstset{frame=tb,
    language=Java,
    aboveskip=3mm,
    belowskip=3mm,
    showstringspaces=false,
    columns=flexible,
    basicstyle={\small\ttfamily},
    numbers=none,
    numberstyle=\tiny\color{gray},
    keywordstyle=\color{blue},
    commentstyle=\color{dkgreen},
    stringstyle=\color{mauve},
    breaklines=true,
    breakatwhitespace=true
    tabsize=3
}

\title{CIS4301 Notes: Database Design}
\author{Ryan Roden-Corrent}
\date{\today}

\begin{document}
\setlength\parindent{0pt}
% Tikz general settings
\tikzstyle{relation} = [diamond, draw, fill=blue!20, text width=4em,
  text badly centered, node distance=3cm, inner sep=0pt]
\tikzstyle{attribute} = [draw, ellipse, fill=red!20, node distance=2.5cm,
  minimum height=2em]
\tikzstyle{entity} = [rectangle, draw, fill=blue!20, text width=5em,
  text centered, minimum height=4em]
\tikzstyle{relation-weak} = [diamond, double, draw, fill=blue!20, text width=4em,
  text badly centered, node distance=3cm, inner sep=0pt]
\tikzstyle{entity-weak} = [rectangle, draw, double, fill=blue!20, text width=5em,
  text centered, minimum height=4em]
\tikzstyle{line} = [draw, -]
\tikzstyle{arrow} = [draw, -latex', thick]
\tikzstyle{arrow-round} = [draw, -), thick]
\maketitle

Note: Some class resources available here:\\
\href{http://cise.ufl.edu/class/cis4301sp14/slides/fd.pdf}{Database Systems
  Textbook}\\
\href{http://cise.ufl.edu/class/cis4301sp14/slides/inference_rules.jpg}{Inference
  Rules}\\
\section{Functional Dependencies and Normalization}
\subsection{Outline}
\begin{itemize}
  \item Informal Design Guidelines for Relation Schemas
  \item Functional Dependencies
  \item normal forms based on primary keys
\end{itemize}

\subsection{Outline}
\begin{itemize}
  \item {
      Levels at which we can discuss \textbf{goodness} of relational schemas
      \begin{itemize}
        \item Logical (conceptual)
        \item Implementation (physical storage)
      \end{itemize}
    }
  \item {
      Approaches to DB design
      \begin{itemize}
        \item Top down: start with large table, break down to details
        \item Bottom Up: start with details, merge to create larger structure
      \end{itemize}
    }
\end{itemize}
\subsection{Informal Design Guidelines}
\subsubsection{Measures of Quality}
\begin{itemize}
  \item make attribute semantics clear
  \item Reduce redundant info in tuples
  \item Reduce NULL values in tuples
  \item Disallow possibility of creating spurious tuples
\end{itemize}

\subsubsection{Natural Join Example}
Schema:\\
EMP\_LOCS(Ename, Plocation)\\
EMP\_PROJ(SSN, Pnumber, Hours, Pnme, Plocation)\\
Want to use natural join to find out how many hours each employee worked at a
location\\
see
\href{http://cise.ufl.edu/class/cis4301sp14/slides/natural_join0.jpg}{this} and
\href{http://cise.ufl.edu/class/cis4301sp14/slides/natural_join0.jpg}{this}
for the textbook problems\\
note that the natural join introduces duplicate information
\textcolor{red}{meaning is ambiguous}.

\textbf{update anomaly:} need to update data in more than one spot. Sometimes
these are unavoidable, so must be \textbf{documented well}.

Triggers are useful in these, but can get messy.

\subsubsection{Redundant Information and Update Anomalies}
\begin{itemize}
  \item {Types: Insertion, Deletion, Modifications}
  \item {Result of storing natural joins of base relations}
  \item {Significant effect on storage stapce}
\end{itemize}

Deletion Anomaly Example:\\
\begin{tabular}{|c|c|}
\hline
Team Name & Player\\
\hline
NULL & 12\\
\hline
\end{tabular}
\begin{tabular}{|c|c|}
\hline
Playerid & pts\\
\hline
12 & 2\\
\hline
\end{tabular}
Player 12 kicked off team, but still exists\\

Insertion Anomaly Example:\\
\begin{tabular}{|c|c|}
\hline
Team Name & Player\\
\hline
Gators & \\
\hline
Louisvile & \\
\hline
? & 12\\
\hline
\end{tabular}
Must know team name before inserting player.

\subsection{Why are NULLs bad?}
\begin{itemize}
  \item {Way to group many attributes together into a "fat" relation}
  \item {Wasted storage space (reserves space in every column for that row)}
\end{itemize}

Suppose an essay relation has a text field that is a char(40000). A null essay
will still reserve this space.

\subsubsection{Guideline 4}
\begin{itemize}
  \item {Design relation schemas to be joined with equality conditions on
      attributes that are appropriately related}
  \item {guarantees no spurious tuples generated}
\end{itemize}

\subsection{Summary}
\begin{itemize}
  \item {anomalies cause redundant work}
  \item {NULL wastes space}
\end{itemize}

\section{Functional Dependencies}
denoted by $X \rightarrow Y$\\
for any two tuples $t_1$ and $t_2$ in $r$ that have $t_1[X] = t_2[X]$, there
must be a mapping $t_1[Y] = t_2[Y]$\\

\begin{tabular}{c|c|c|c|c}
  & W & X & Y & Z\\
  \hline
  $t_1$ &  & billy d & gators & \\
  \hline
  $t_2$ & & billy d & gators & \\
  \hline
\end{tabular}

\begin{itemize}
  \item {formal tool for analysis of relational schemas}
  \item {detect and describe some of the above problems in precise terms}
  \item {theory of functinal dependency}
\end{itemize}

\subsection{normal forms based on primary keys}
\begin{itemize}
  \item {normalization process}
  \item {Approaches for relational schema design}
\end{itemize}
\begin{itemize}
  \item {
      takes a relation schema through a series of tests
      \begin{itemize}
        \item {certify whether it satisfies a certain normal form}
        \item {proceeds in top-down fashion}
      \end{itemize}
    }
  \item {ideally only want functional dependencies on primary keys (but there
      is some leeway)}
\end{itemize}

\subsection{Normalization of Relations}
\begin{itemize}
  \item {
      Nonadditive join Property
      \begin{itemize}
        \item {Extrememly critical}
      \end{itemize}
    }
  \item {
      Dependency preservation property
      \begin{itemize}
        \item {sometimes sacrificed for other factors}
      \end{itemize}
    }
\end{itemize}
\subsubsection{Keys and Attributes Participating in Keys}
\begin{itemize}
  \item {definition of \textbf{superset} and \textbf{key}}
  \item {Candidate Key}
  \item {
    If more than one key in relation schema
    \begin{itemize}
      \item {one is primary}
      \item {others are secondary}
    \end{itemize}
  }
\end{itemize}
a \textbf{superkey} determines all of the attributes in a set.

\subsection{Armstrong's axioms}
\textbf{Reflexive:}

\textbf{Augmentation:}\\
\textbf{Transitive:}
\begin{math}
  S \rightarrow R, R \rightarrow T => S \rightarrow T
\end{math}

\textbf{Decomposition:}
\begin{math}
  X \rightarrow YX => X \rightarrow Y, X \rightarrow Z
\end{math}

\textbf{Union:}

\textbf{PseudoTransitivity:}
\begin{math}
  X \rightarrow Y, WY \rightarrow Z, WX \rightarrow Z
\end{math}
\subsubsection{Example}
Decomposition:
\begin{math}
  pname \rightarrow pts, team\\
  pname \rightarrow pts\\
  pname \rightarrow team
\end{math}

\subsection{Closure}
Full set of functional dependencies that can  created given an initial set of
functional dependencies, denoted \textbf{F+}.

\subsection{Properties}
\begin{description}
  \item[Non-additive join] shouldn't lose or gain information when performing a
    join
  \item[Dependency Preservation] dependencies should hold through decomposition
\end{description}
\subsection{Practical Normal Forms}
\begin{itemize}
  \item {Resulting designs are high quality}
  \item {usually pay attention to normaliation up to 3NF, BCNF, of sometimes 4NF}
\end{itemize}
\subsubsection{First Normal Form}
\begin{itemize}
  \item {only attribute values permitted are single (atomic) values}
\end{itemize}
Example:\\
\begin{tabular}{|c|c|c|}
  \hline
students & age & hometown \\
  \hline
patriczy & 21 & \{Jax, Warner Robins\} \\
  \hline
\end{tabular}\\
This violates 1NF as hometown is a collection, and therefore not atomic. Could
be solved with hometown mapping table:\\
\begin{tabular}{|c|c|c|}
  \hline
 pyoung & 21 & jax\\
  \hline
 pyoung & 21 & Warner Robins\\
  \hline
\end{tabular}

\subsubsection{Second Normal Form}
\begin{itemize}
  \item {based on \textbf{full functional dependency}}
  \item {nonprime attribtes are associated with only part of a primary key on
      which they are functionally dependent}
\end{itemize}
Example:\\
\begin{tabular}{|c|c|c|c|}
  \hline
 movie name & star  & film type & capacity\\
  \hline
 ... & ... & ... & ... \\
  \hline
\end{tabular}\\
Functional dependencies:\\
$moviename \rightarrow star$\\
$filmtype \rightarrow Capacity$\\
Could split into two tables:
\begin{tabular}{|c|c|}
  \hline
 movie name & star  \\
  \hline
 ... & ... \\
  \hline
\end{tabular}
\begin{tabular}{|c|c|}
  \hline
 filmtype & capacity  \\
  \hline
 ... & ... \\
  \hline
\end{tabular}\\
Dependencies:\\
$moviename \rightarrow star$\\
$filmtype \rightarrow capacity$\\
$star \rightarrow filmtype$\\
Transitive rule relates filmtype to moviename.
\subsubsection{Third Normal Form}
\begin{itemize}
  \item {Be in first normal form}
  \item {Everything dependent on primary key}
  \item {Dependencies on primary key must not be transitive}
\end{itemize}
\end{document}
